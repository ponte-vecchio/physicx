\documentclass{article}
\usepackage[colorlinks=true]{hyperref}
\makeatletter
% Constants
% May be redundant
\gdef\fn@s{s} %square [x]
\gdef\fn@c{c} %braces {x}
\gdef\fn@v{v} %vert |x|
\gdef\fn@abs{abs} % dvert ||x||
\gdef\fn@a{a} % angled <x>
\gdef\fn@r{r} %round (x)
\gdef\br@Bigg{Bg}
\gdef\br@bigg{bg}
\gdef\br@Big{B}
\gdef\br@big{b}

\newcommand{\br@typarse}[2][r]{% Parse the type of brackets to output
	\def\fn@arg{#1}
	\let\fn@L( \let\fn@R)
	\ifx\fn@arg\fn@r \relax
	\else
		\ifx\fn@arg\fn@s	\let\fn@L[	\let\fn@R]
		\else
			\ifx\fn@arg\fn@v	\let\fn@L|	\let\fn@R|
			\else
				\ifx\fn@arg\fn@c	\let\fn@L\{	\let\fn@R\}
				\else
					\ifx\fn@arg\fn@a	\let\fn@L\langle \let\fn@R\rangle
					\else
						\ifx\fn@arg\fn@abs \let\fn@L\| \let\fn@R\|
						\else\errmessage{%
							Invalid option for bracket type!
							Supported types are:[r, s, v, c, a, abs]%
						}
						\fi
					\fi
				\fi
			\fi
		\fi
	\fi
	\fn@L#2\fn@R%
}
\newcommand{\br@szparse}{% Parse the size of brackets to output
	\@ifstar{\br@szparse@nolr}{\br@szparse@lr}%
}
\newcommand{\br@szparse@lr}[4]{%
	% #1: arg for bracket size
	% #2: left-side bracket type
	% #3: content
	% #4: right-side bracket type
	\def\sz@arg{#1}
	\ifx\sz@arg\empty \left#2 #3 \right#4
	\else
	\ifx\sz@arg\br@Bigg \Biggl#2 #3 \Biggr#4
	\else
	\ifx\sz@arg\br@bigg \biggl#2 #3 \biggr#4
	\else
	\ifx\sz@arg\br@Big \Bigl#2 #3 \Bigr#4
	\else
	\ifx\sz@arg\br@big \bigl#2 #3 \bigr#4
	\else
	\expandafter\errmessage{%
		Invalid option for bracket size!
		Available options are: [\br@Bigg,\br@bigg,\br@Big,\br@big, or None]%
	}
	\fi
	\fi
	\fi
	\fi
	\fi	
}
\newcommand{\br@szparse@nolr}[4]{%
	% Same as szparse@lr, but for big instead of big[lr] etc.
	\def\sz@arg{#1}
	\ifx\sz@arg\empty \left#2 #3 \right#4
	\else
		\ifx\sz@arg\br@Bigg \Bigg#2 #3 \Bigg#4
		\else
			\ifx\sz@arg\br@bigg \bigg#2 #3 \bigg#4
			\else
				\ifx\sz@arg\br@Big \Big#2 #3 \Big#4
				\else
					\ifx\sz@arg\br@big \big#2 #3 \big#4
					\else
					\expandafter\errmessage{%
						Invalid option for bracket size!
						Available options are: [\br@Bigg,\br@bigg,\br@Big,\br@big, or None]%
					}
					\fi
				\fi
			\fi
		\fi
	\fi
}

% To use \d, we must move accentchar \d to \udot
\let\udot\d
\let\d\undefined

% Simple dx with exponents
\newcommand{\d}{\ensuremath{\@ifstar{\d@exp}{\d@one}}}
\long\def\d@one#1{\mathrm{d}#1}
\newcommand{\d@exp}[2][2]{\mathrm{d}^{#1}#2}

% Partial dx with exponents
\newcommand{\pd}{\ensuremath{\@ifstar{\pd@exp}{\pd@one}}}
\long\def\pd@one#1{\partial #1}
\newcommand{\pd@exp}[2][2]{\partial^{#1}#2}

% Single dy/dx with exponents
\newcommand{\dd}{\ensuremath{\@ifstar{\dd@exp}{\dd@one}}}
\long\def\dd@one#1#2{\frac{\d{#1}}{\d{#2}}\,}
\newcommand{\dd@exp}[3][2]{\frac{\d*[#1]#2}{\d#3^{#1}}}

% Partial dy/dx with exponents
\newcommand{\pdd}{\ensuremath{\@ifstar{\pdd@exp}{\pdd@one}}}
\long\def\pdd@one#1#2{\frac{\pd{#1}}{\pd{#2}}}
\newcommand{\pdd@exp}[3][2]{\frac{\pd*[#1]#2}{\pd#3^{#1}}}

% Functions
\newcommand{\fn}{\ensuremath{\@ifstar{\fn@exp}{\fn@one}}}
\newcommand{\fn@one}[3][r]{#2\br@typarse[#1]{#3}}
\newcommand{\fn@exp}[4][r]{#2^{#3}\br@typarse[#1]{#4}}

% Named things
\newcommand{\leibniz}[3]{\ensuremath{#1'\kern-.1667em\brac[b]{ #2'\kern-.1667em \brac{#3} }\cdot#2'\kern-.1667em\brac{#3}}}



% Brackets
\newcommand{\brac}[2][]{\br@szparse{#1}{(}{#2}{)}}
\newcommand{\sbrac}[2][]{\br@szparse{#1}{[}{#2}{]}}
\newcommand{\cbrac}[2][]{\br@szparse{#1}{\{}{#2}{\}}}
\newcommand{\vbrac}[2][]{\br@szparse*{#1}{|}{#2}{|}}
\newcommand{\abs}[2][]{\br@szparse*{#1}{\|}{#2}{\|}}
\newcommand{\abrac}[2][]{\br@szparse{#1}{\langle}{#2}{\rangle}}
\newcommand{\floor}[2][]{\br@szparse{#1}{\lfloor}{#2}{\rfloor}}
\newcommand{\ceil}[2][]{\br@szparse{#1}{\lceil}{#2}{\rceil}}

\makeatother


\begin{document}
\title{The {\tt physicx} package}
\author{Leothelion\\ {\tt zkogdxdkur\kern.1667em(\kern-.067em at\kern-.067em)\kern.1667emp.monash.edu}}
\date{\today}
\maketitle

\tableofcontents\pagebreak
\section{Before You Begin}
\subsection{The purpose of this package}

The goal of this package was to make writing physics simple and intuitive with its syntax in line with the conventions of \LaTeX\ and \TeX\ thereof.

\subsection{How to use {\tt physicx} in your \LaTeX\ or \TeX\ document}

For \LaTeX\ users, simply add {\verb|\usepackage{physicx}|} to your preamble as per the standard procedure for loading packages:

\noindent\hrulefill
\begin{verbatim}
\documentclass[ ... ]{ ... }

\usepackage{physicx}

\begin{document}
	
...

\end{document}
\end{verbatim}
\hrulefill\vskip1em

For \TeX\ users, load the package by using the \verb|\input| command such that:

\noindent\hrulefill
\begin{verbatim}
...

\input physicx

...
\end{verbatim}
\hrulefill\pagebreak

\section{List of Commands}
\everymath{\displaystyle}
\subsection{Brackets}
% \ensuremath{\foreach \j in {0,1,...,7175}{\fn[s]{f}{x,y}=\frac{\fn{g}{x,y}\pdd{f}{x}-\fn{f}{x,y}\pdd{g}{x}}{\sbrac[b]{\fn{g}{x,y}}^2}}}
\vbox{
	\ialign{\tabskip=2em
		% Command | Examples | Description
		\hss\tt #&
		\hss$#$\hss&
		\hss$#$\hss&
		# \hss\cr
		\verb+\brac+ & \brac{a}& \brac{\hbox{\Huge g}}&round brackets\cr
		\verb+\sbrac+&\sbrac{a}&\sbrac{\hbox{\Huge g}}&square brackets\cr
		\verb+\cbrac+&\cbrac{a}&\cbrac{\hbox{\Huge g}}&curly brackets\cr
		\verb+\vbrac+&\vbrac{a}&\vbrac{\hbox{\Huge g}}&vertical brackets\cr
		\verb+\abs+  &  \abs{a}&  \abs{\hbox{\Huge g}}&absolute\cr
	}
}

\end{document}
